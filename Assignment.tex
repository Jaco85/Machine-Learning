\documentclass[]{article}
\usepackage{lmodern}
\usepackage{amssymb,amsmath}
\usepackage{ifxetex,ifluatex}
\usepackage{fixltx2e} % provides \textsubscript
\ifnum 0\ifxetex 1\fi\ifluatex 1\fi=0 % if pdftex
  \usepackage[T1]{fontenc}
  \usepackage[utf8]{inputenc}
\else % if luatex or xelatex
  \ifxetex
    \usepackage{mathspec}
  \else
    \usepackage{fontspec}
  \fi
  \defaultfontfeatures{Ligatures=TeX,Scale=MatchLowercase}
\fi
% use upquote if available, for straight quotes in verbatim environments
\IfFileExists{upquote.sty}{\usepackage{upquote}}{}
% use microtype if available
\IfFileExists{microtype.sty}{%
\usepackage{microtype}
\UseMicrotypeSet[protrusion]{basicmath} % disable protrusion for tt fonts
}{}
\usepackage[margin=1in]{geometry}
\usepackage{hyperref}
\hypersetup{unicode=true,
            pdftitle={Assigment},
            pdfauthor={Jaco},
            pdfborder={0 0 0},
            breaklinks=true}
\urlstyle{same}  % don't use monospace font for urls
\usepackage{color}
\usepackage{fancyvrb}
\newcommand{\VerbBar}{|}
\newcommand{\VERB}{\Verb[commandchars=\\\{\}]}
\DefineVerbatimEnvironment{Highlighting}{Verbatim}{commandchars=\\\{\}}
% Add ',fontsize=\small' for more characters per line
\usepackage{framed}
\definecolor{shadecolor}{RGB}{248,248,248}
\newenvironment{Shaded}{\begin{snugshade}}{\end{snugshade}}
\newcommand{\KeywordTok}[1]{\textcolor[rgb]{0.13,0.29,0.53}{\textbf{#1}}}
\newcommand{\DataTypeTok}[1]{\textcolor[rgb]{0.13,0.29,0.53}{#1}}
\newcommand{\DecValTok}[1]{\textcolor[rgb]{0.00,0.00,0.81}{#1}}
\newcommand{\BaseNTok}[1]{\textcolor[rgb]{0.00,0.00,0.81}{#1}}
\newcommand{\FloatTok}[1]{\textcolor[rgb]{0.00,0.00,0.81}{#1}}
\newcommand{\ConstantTok}[1]{\textcolor[rgb]{0.00,0.00,0.00}{#1}}
\newcommand{\CharTok}[1]{\textcolor[rgb]{0.31,0.60,0.02}{#1}}
\newcommand{\SpecialCharTok}[1]{\textcolor[rgb]{0.00,0.00,0.00}{#1}}
\newcommand{\StringTok}[1]{\textcolor[rgb]{0.31,0.60,0.02}{#1}}
\newcommand{\VerbatimStringTok}[1]{\textcolor[rgb]{0.31,0.60,0.02}{#1}}
\newcommand{\SpecialStringTok}[1]{\textcolor[rgb]{0.31,0.60,0.02}{#1}}
\newcommand{\ImportTok}[1]{#1}
\newcommand{\CommentTok}[1]{\textcolor[rgb]{0.56,0.35,0.01}{\textit{#1}}}
\newcommand{\DocumentationTok}[1]{\textcolor[rgb]{0.56,0.35,0.01}{\textbf{\textit{#1}}}}
\newcommand{\AnnotationTok}[1]{\textcolor[rgb]{0.56,0.35,0.01}{\textbf{\textit{#1}}}}
\newcommand{\CommentVarTok}[1]{\textcolor[rgb]{0.56,0.35,0.01}{\textbf{\textit{#1}}}}
\newcommand{\OtherTok}[1]{\textcolor[rgb]{0.56,0.35,0.01}{#1}}
\newcommand{\FunctionTok}[1]{\textcolor[rgb]{0.00,0.00,0.00}{#1}}
\newcommand{\VariableTok}[1]{\textcolor[rgb]{0.00,0.00,0.00}{#1}}
\newcommand{\ControlFlowTok}[1]{\textcolor[rgb]{0.13,0.29,0.53}{\textbf{#1}}}
\newcommand{\OperatorTok}[1]{\textcolor[rgb]{0.81,0.36,0.00}{\textbf{#1}}}
\newcommand{\BuiltInTok}[1]{#1}
\newcommand{\ExtensionTok}[1]{#1}
\newcommand{\PreprocessorTok}[1]{\textcolor[rgb]{0.56,0.35,0.01}{\textit{#1}}}
\newcommand{\AttributeTok}[1]{\textcolor[rgb]{0.77,0.63,0.00}{#1}}
\newcommand{\RegionMarkerTok}[1]{#1}
\newcommand{\InformationTok}[1]{\textcolor[rgb]{0.56,0.35,0.01}{\textbf{\textit{#1}}}}
\newcommand{\WarningTok}[1]{\textcolor[rgb]{0.56,0.35,0.01}{\textbf{\textit{#1}}}}
\newcommand{\AlertTok}[1]{\textcolor[rgb]{0.94,0.16,0.16}{#1}}
\newcommand{\ErrorTok}[1]{\textcolor[rgb]{0.64,0.00,0.00}{\textbf{#1}}}
\newcommand{\NormalTok}[1]{#1}
\usepackage{graphicx,grffile}
\makeatletter
\def\maxwidth{\ifdim\Gin@nat@width>\linewidth\linewidth\else\Gin@nat@width\fi}
\def\maxheight{\ifdim\Gin@nat@height>\textheight\textheight\else\Gin@nat@height\fi}
\makeatother
% Scale images if necessary, so that they will not overflow the page
% margins by default, and it is still possible to overwrite the defaults
% using explicit options in \includegraphics[width, height, ...]{}
\setkeys{Gin}{width=\maxwidth,height=\maxheight,keepaspectratio}
\IfFileExists{parskip.sty}{%
\usepackage{parskip}
}{% else
\setlength{\parindent}{0pt}
\setlength{\parskip}{6pt plus 2pt minus 1pt}
}
\setlength{\emergencystretch}{3em}  % prevent overfull lines
\providecommand{\tightlist}{%
  \setlength{\itemsep}{0pt}\setlength{\parskip}{0pt}}
\setcounter{secnumdepth}{0}
% Redefines (sub)paragraphs to behave more like sections
\ifx\paragraph\undefined\else
\let\oldparagraph\paragraph
\renewcommand{\paragraph}[1]{\oldparagraph{#1}\mbox{}}
\fi
\ifx\subparagraph\undefined\else
\let\oldsubparagraph\subparagraph
\renewcommand{\subparagraph}[1]{\oldsubparagraph{#1}\mbox{}}
\fi

%%% Use protect on footnotes to avoid problems with footnotes in titles
\let\rmarkdownfootnote\footnote%
\def\footnote{\protect\rmarkdownfootnote}

%%% Change title format to be more compact
\usepackage{titling}

% Create subtitle command for use in maketitle
\newcommand{\subtitle}[1]{
  \posttitle{
    \begin{center}\large#1\end{center}
    }
}

\setlength{\droptitle}{-2em}

  \title{Assigment}
    \pretitle{\vspace{\droptitle}\centering\huge}
  \posttitle{\par}
    \author{Jaco}
    \preauthor{\centering\large\emph}
  \postauthor{\par}
      \predate{\centering\large\emph}
  \postdate{\par}
    \date{25-1-2019}


\begin{document}
\maketitle

\subsection{Introduction}\label{introduction}

Using devices such as Jawbone Up, Nike FuelBand, and Fitbit it is now
possible to collect a large amount of data about personal activity
relatively inexpensively. These type of devices are part of the
quantified self movement -- a group of enthusiasts who take measurements
about themselves regularly to improve their health, to find patterns in
their behavior, or because they are tech geeks. One thing that people
regularly do is quantify how much of a particular activity they do, but
they rarely quantify how well they do it. In this project, your goal
will be to use data from accelerometers on the belt, forearm, arm, and
dumbell of 6 participants. They were asked to perform barbell lifts
correctly and incorrectly in 5 different ways.

More information is available from the website here:
\url{http://web.archive.org/web/20161224072740/http:/groupware.les.inf.puc-rio.br/har}
(see the section on the Weight Lifting Exercise Dataset).

Data

The training data for this project are available here:
\url{https://d396qusza40orc.cloudfront.net/predmachlearn/pml-training.csv}

The test data are available here:
\url{https://d396qusza40orc.cloudfront.net/predmachlearn/pml-testing.csv}

The data for this project come from this source:
\url{http://web.archive.org/web/20161224072740/http:/groupware.les.inf.puc-rio.br/har}.
If you use the document you create for this class for any purpose please
cite them as they have been very generous in allowing their data to be
used for this kind of assignment.

The goal of your project is to predict the manner in which they did the
exercise. This is the ``classe'' variable in the training set. You may
use any of the other variables to predict with. You should create a
report describing how you built your model, how you used cross
validation, what you think the expected out of sample error is, and why
you made the choices you did. You will also use your prediction model to
predict 20 different test cases.

\subsection{Importing Data:}\label{importing-data}

\begin{Shaded}
\begin{Highlighting}[]
\KeywordTok{setwd}\NormalTok{(}\StringTok{"/Users/jaco/Documents/datasciencecoursera/Course 8_Machine Learning/Week 4"}\NormalTok{)}
\NormalTok{dataset <-}\StringTok{ }\KeywordTok{read.csv}\NormalTok{(}\StringTok{'./pml-training.csv'}\NormalTok{, }\DataTypeTok{header=}\NormalTok{T)}
\NormalTok{test<-}\StringTok{ }\KeywordTok{read.csv}\NormalTok{(}\StringTok{'./pml-testing.csv'}\NormalTok{, }\DataTypeTok{header=}\NormalTok{T)}
\KeywordTok{dim}\NormalTok{(dataset)}
\end{Highlighting}
\end{Shaded}

\begin{verbatim}
## [1] 19622   160
\end{verbatim}

\subsection{Exploration and
preprocessing:}\label{exploration-and-preprocessing}

\paragraph{Above we see that there are 160 variables counting 19622
records. Let's explore te quality of the
data:}\label{above-we-see-that-there-are-160-variables-counting-19622-records.-lets-explore-te-quality-of-the-data}

\begin{Shaded}
\begin{Highlighting}[]
\NormalTok{na_count <-}\KeywordTok{sapply}\NormalTok{(dataset, }\ControlFlowTok{function}\NormalTok{(y) }\KeywordTok{sum}\NormalTok{(}\KeywordTok{length}\NormalTok{(}\KeywordTok{which}\NormalTok{(}\KeywordTok{is.na}\NormalTok{(y)))))}
\NormalTok{na_count_df <-}\StringTok{ }\KeywordTok{data.frame}\NormalTok{(na_count}\OperatorTok{>}\DecValTok{0}\NormalTok{)}
\KeywordTok{sum}\NormalTok{(na_count_df}\OperatorTok{$}\NormalTok{na_count)}
\end{Highlighting}
\end{Shaded}

\begin{verbatim}
## [1] 67
\end{verbatim}

\paragraph{67 columns have NA's, we will exclude these from the
dataset;}\label{columns-have-nas-we-will-exclude-these-from-the-dataset}

\begin{Shaded}
\begin{Highlighting}[]
\NormalTok{dataset_clean <-}\StringTok{ }\NormalTok{dataset[,}\KeywordTok{which}\NormalTok{(na_count }\OperatorTok{==}\StringTok{ }\DecValTok{0}\NormalTok{)]}
\KeywordTok{dim}\NormalTok{(dataset_clean)}
\end{Highlighting}
\end{Shaded}

\begin{verbatim}
## [1] 19622    93
\end{verbatim}

\paragraph{93 columns are left for the analysis, now we will exclude the
columns with zero or near zero
variance:}\label{columns-are-left-for-the-analysis-now-we-will-exclude-the-columns-with-zero-or-near-zero-variance}

\begin{Shaded}
\begin{Highlighting}[]
\KeywordTok{library}\NormalTok{(caret)}
\end{Highlighting}
\end{Shaded}

\begin{verbatim}
## Loading required package: lattice
\end{verbatim}

\begin{verbatim}
## Loading required package: ggplot2
\end{verbatim}

\begin{Shaded}
\begin{Highlighting}[]
\NormalTok{NZV <-}\StringTok{ }\KeywordTok{nearZeroVar}\NormalTok{(dataset_clean)}
\NormalTok{dataset_clean <-}\StringTok{ }\NormalTok{dataset_clean[}\OperatorTok{-}\NormalTok{NZV]}
\KeywordTok{dim}\NormalTok{(dataset_clean)}
\end{Highlighting}
\end{Shaded}

\begin{verbatim}
## [1] 19622    59
\end{verbatim}

\paragraph{Finally we exclude the first 6 descriptive columns that
can'st be predicters (name / timestamps /
num\_window):}\label{finally-we-exclude-the-first-6-descriptive-columns-that-canst-be-predicters-name-timestamps-num_window}

\begin{Shaded}
\begin{Highlighting}[]
\NormalTok{dataset_clean <-}\StringTok{ }\NormalTok{dataset_clean[,}\DecValTok{7}\OperatorTok{:}\DecValTok{59}\NormalTok{]}
\KeywordTok{dim}\NormalTok{(dataset_clean)}
\end{Highlighting}
\end{Shaded}

\begin{verbatim}
## [1] 19622    53
\end{verbatim}

\paragraph{53 columns / potential predicters are left to train our
machine learning model, we split this cleanes dataset in a training
(80\%) and validation set (20\%) for performing the cross
validation:}\label{columns-potential-predicters-are-left-to-train-our-machine-learning-model-we-split-this-cleanes-dataset-in-a-training-80-and-validation-set-20-for-performing-the-cross-validation}

\begin{Shaded}
\begin{Highlighting}[]
\NormalTok{inTrain <-}\StringTok{ }\KeywordTok{createDataPartition}\NormalTok{(}\DataTypeTok{y=}\NormalTok{dataset_clean}\OperatorTok{$}\NormalTok{classe, }\DataTypeTok{p=}\FloatTok{0.8}\NormalTok{, }\DataTypeTok{list=}\OtherTok{FALSE}\NormalTok{)}
\NormalTok{Training <-}\StringTok{ }\NormalTok{dataset_clean[inTrain, ]}
\NormalTok{Validation <-}\StringTok{ }\NormalTok{dataset_clean[}\OperatorTok{-}\NormalTok{inTrain, ]}
\end{Highlighting}
\end{Shaded}

\paragraph{Ok, we're finished the preprocessing proces, the datasets can
now be used for building ML
models}\label{ok-were-finished-the-preprocessing-proces-the-datasets-can-now-be-used-for-building-ml-models}

\subsection{Machine Learning models}\label{machine-learning-models}

\paragraph{\texorpdfstring{We will predict the classification
``classe'', where we try 2 algorithmns we learned in the course: the
Decision Tree and Random Forest Algorithm. We first have a look how the
decision tree works
out:}{We will predict the classification classe, where we try 2 algorithmns we learned in the course: the Decision Tree and Random Forest Algorithm. We first have a look how the decision tree works out:}}\label{we-will-predict-the-classification-classe-where-we-try-2-algorithmns-we-learned-in-the-course-the-decision-tree-and-random-forest-algorithm.-we-first-have-a-look-how-the-decision-tree-works-out}

\begin{Shaded}
\begin{Highlighting}[]
\KeywordTok{library}\NormalTok{(rpart) }
\KeywordTok{library}\NormalTok{(rpart.plot)}

\CommentTok{#Classification Tree}
\KeywordTok{set.seed}\NormalTok{(}\DecValTok{1234}\NormalTok{)}
\NormalTok{mod_dt <-}\StringTok{ }\KeywordTok{rpart}\NormalTok{(classe }\OperatorTok{~}\StringTok{ }\NormalTok{., }\DataTypeTok{data=}\NormalTok{Training, }\DataTypeTok{method=}\StringTok{"class"}\NormalTok{)}

\CommentTok{#Plot Decision Tree}
\KeywordTok{rpart.plot}\NormalTok{(mod_dt, }\DataTypeTok{main=}\StringTok{"Classification Tree"}\NormalTok{, }\DataTypeTok{extra=}\DecValTok{102}\NormalTok{, }\DataTypeTok{under=}\OtherTok{TRUE}\NormalTok{, }\DataTypeTok{faclen=}\DecValTok{0}\NormalTok{)}
\end{Highlighting}
\end{Shaded}

\includegraphics{Assignment_files/figure-latex/unnamed-chunk-6-1.pdf}

\begin{Shaded}
\begin{Highlighting}[]
\NormalTok{pred_dt_train <-}\StringTok{ }\KeywordTok{predict}\NormalTok{(mod_dt, Training, }\DataTypeTok{type =} \StringTok{"class"}\NormalTok{)}
\NormalTok{CM_dt <-}\StringTok{ }\KeywordTok{confusionMatrix}\NormalTok{(Training}\OperatorTok{$}\NormalTok{classe, pred_dt_train)}
\NormalTok{CM_dt}
\end{Highlighting}
\end{Shaded}

\begin{verbatim}
## Confusion Matrix and Statistics
## 
##           Reference
## Prediction    A    B    C    D    E
##          A 3702  119  117  467   59
##          B  415 1687  339  508   89
##          C   52  236 2062  320   68
##          D  111  225  359 1770  108
##          E   38  219  348  433 1848
## 
## Overall Statistics
##                                           
##                Accuracy : 0.7051          
##                  95% CI : (0.6979, 0.7122)
##     No Information Rate : 0.275           
##     P-Value [Acc > NIR] : < 2.2e-16       
##                                           
##                   Kappa : 0.6283          
##  Mcnemar's Test P-Value : < 2.2e-16       
## 
## Statistics by Class:
## 
##                      Class: A Class: B Class: C Class: D Class: E
## Sensitivity            0.8573   0.6786   0.6394   0.5060   0.8508
## Specificity            0.9330   0.8978   0.9458   0.9342   0.9233
## Pos Pred Value         0.8293   0.5553   0.7531   0.6879   0.6403
## Neg Pred Value         0.9452   0.9369   0.9103   0.8684   0.9747
## Prevalence             0.2750   0.1584   0.2054   0.2228   0.1384
## Detection Rate         0.2358   0.1075   0.1313   0.1127   0.1177
## Detection Prevalence   0.2843   0.1935   0.1744   0.1639   0.1838
## Balanced Accuracy      0.8952   0.7882   0.7926   0.7201   0.8870
\end{verbatim}

\begin{Shaded}
\begin{Highlighting}[]
\NormalTok{Accuracy_dt <-}\StringTok{ }\KeywordTok{round}\NormalTok{(CM_dt}\OperatorTok{$}\NormalTok{overall[}\DecValTok{1}\NormalTok{]}\OperatorTok{*}\DecValTok{100}\NormalTok{,}\DecValTok{2}\NormalTok{)}
\end{Highlighting}
\end{Shaded}

\paragraph{Results of using the Decision Tree algorithm shows a accuracy
of 70.51\%. Let's try the Random Forest
algorithm:}\label{results-of-using-the-decision-tree-algorithm-shows-a-accuracy-of-70.51.-lets-try-the-random-forest-algorithm}

\begin{Shaded}
\begin{Highlighting}[]
\KeywordTok{library}\NormalTok{(randomForest)}
\end{Highlighting}
\end{Shaded}

\begin{verbatim}
## randomForest 4.6-14
\end{verbatim}

\begin{verbatim}
## Type rfNews() to see new features/changes/bug fixes.
\end{verbatim}

\begin{verbatim}
## 
## Attaching package: 'randomForest'
\end{verbatim}

\begin{verbatim}
## The following object is masked from 'package:ggplot2':
## 
##     margin
\end{verbatim}

\begin{Shaded}
\begin{Highlighting}[]
\NormalTok{mod_rf <-}\StringTok{ }\KeywordTok{randomForest}\NormalTok{(classe }\OperatorTok{~}\StringTok{ }\NormalTok{., }\DataTypeTok{data =}\NormalTok{ Training)}

\NormalTok{pred_rf_train <-}\StringTok{ }\KeywordTok{predict}\NormalTok{(mod_rf, Training)}
\NormalTok{CM_rf <-}\StringTok{ }\KeywordTok{confusionMatrix}\NormalTok{(Training}\OperatorTok{$}\NormalTok{classe, pred_rf_train)}
\NormalTok{CM_rf}
\end{Highlighting}
\end{Shaded}

\begin{verbatim}
## Confusion Matrix and Statistics
## 
##           Reference
## Prediction    A    B    C    D    E
##          A 4464    0    0    0    0
##          B    0 3038    0    0    0
##          C    0    0 2738    0    0
##          D    0    0    0 2573    0
##          E    0    0    0    0 2886
## 
## Overall Statistics
##                                      
##                Accuracy : 1          
##                  95% CI : (0.9998, 1)
##     No Information Rate : 0.2843     
##     P-Value [Acc > NIR] : < 2.2e-16  
##                                      
##                   Kappa : 1          
##  Mcnemar's Test P-Value : NA         
## 
## Statistics by Class:
## 
##                      Class: A Class: B Class: C Class: D Class: E
## Sensitivity            1.0000   1.0000   1.0000   1.0000   1.0000
## Specificity            1.0000   1.0000   1.0000   1.0000   1.0000
## Pos Pred Value         1.0000   1.0000   1.0000   1.0000   1.0000
## Neg Pred Value         1.0000   1.0000   1.0000   1.0000   1.0000
## Prevalence             0.2843   0.1935   0.1744   0.1639   0.1838
## Detection Rate         0.2843   0.1935   0.1744   0.1639   0.1838
## Detection Prevalence   0.2843   0.1935   0.1744   0.1639   0.1838
## Balanced Accuracy      1.0000   1.0000   1.0000   1.0000   1.0000
\end{verbatim}

\begin{Shaded}
\begin{Highlighting}[]
\NormalTok{Accuracy_rf <-}\StringTok{ }\KeywordTok{round}\NormalTok{(CM_rf}\OperatorTok{$}\NormalTok{overall[}\DecValTok{1}\NormalTok{]}\OperatorTok{*}\DecValTok{100}\NormalTok{,}\DecValTok{2}\NormalTok{)}
\end{Highlighting}
\end{Shaded}

\paragraph{Results of using the Random Forest algorithm shows a accuracy
of 100\%. Let's validate this model on validation dataset we just
created:}\label{results-of-using-the-random-forest-algorithm-shows-a-accuracy-of-100.-lets-validate-this-model-on-validation-dataset-we-just-created}

\paragraph{Cross Validation of the Random Forest
model:}\label{cross-validation-of-the-random-forest-model}

\begin{Shaded}
\begin{Highlighting}[]
\NormalTok{pred_rf_val <-}\StringTok{ }\KeywordTok{predict}\NormalTok{(mod_rf, Validation)}
\NormalTok{CM_rf_val <-}\StringTok{ }\KeywordTok{confusionMatrix}\NormalTok{(Validation}\OperatorTok{$}\NormalTok{classe, pred_rf_val)}
\NormalTok{CM_rf_val}
\end{Highlighting}
\end{Shaded}

\begin{verbatim}
## Confusion Matrix and Statistics
## 
##           Reference
## Prediction    A    B    C    D    E
##          A 1115    1    0    0    0
##          B    1  757    1    0    0
##          C    0    2  681    1    0
##          D    0    0    8  634    1
##          E    0    0    0    0  721
## 
## Overall Statistics
##                                           
##                Accuracy : 0.9962          
##                  95% CI : (0.9937, 0.9979)
##     No Information Rate : 0.2845          
##     P-Value [Acc > NIR] : < 2.2e-16       
##                                           
##                   Kappa : 0.9952          
##  Mcnemar's Test P-Value : NA              
## 
## Statistics by Class:
## 
##                      Class: A Class: B Class: C Class: D Class: E
## Sensitivity            0.9991   0.9961   0.9870   0.9984   0.9986
## Specificity            0.9996   0.9994   0.9991   0.9973   1.0000
## Pos Pred Value         0.9991   0.9974   0.9956   0.9860   1.0000
## Neg Pred Value         0.9996   0.9991   0.9972   0.9997   0.9997
## Prevalence             0.2845   0.1937   0.1759   0.1619   0.1840
## Detection Rate         0.2842   0.1930   0.1736   0.1616   0.1838
## Detection Prevalence   0.2845   0.1935   0.1744   0.1639   0.1838
## Balanced Accuracy      0.9994   0.9977   0.9930   0.9978   0.9993
\end{verbatim}

\begin{Shaded}
\begin{Highlighting}[]
\NormalTok{Accuracy_rf_val <-}\StringTok{ }\KeywordTok{round}\NormalTok{(CM_rf_val}\OperatorTok{$}\NormalTok{overall[}\DecValTok{1}\NormalTok{]}\OperatorTok{*}\DecValTok{100}\NormalTok{,}\DecValTok{2}\NormalTok{)}
\end{Highlighting}
\end{Shaded}

\paragraph{Cross Validation results of the trained Random Forest model
shows a accuracy of 99.62\%. (Out-of-Sample error is
0.38\%).}\label{cross-validation-results-of-the-trained-random-forest-model-shows-a-accuracy-of-99.62.-out-of-sample-error-is-0.38.}

\subsection{Conclusion:}\label{conclusion}

\paragraph{\texorpdfstring{In this case the random forest algorithm
outperformed the decision tree algoritm. We were able to predict the
``classe'' with a high accuracy of 99.62\%. The 20 most important
features
are:}{In this case the random forest algorithm outperformed the decision tree algoritm. We were able to predict the classe with a high accuracy of 99.62\%. The 20 most important features are:}}\label{in-this-case-the-random-forest-algorithm-outperformed-the-decision-tree-algoritm.-we-were-able-to-predict-the-classe-with-a-high-accuracy-of-99.62.-the-20-most-important-features-are}

\begin{Shaded}
\begin{Highlighting}[]
\NormalTok{Feature_imp <-}\StringTok{ }\KeywordTok{importance}\NormalTok{(mod_rf)}
\NormalTok{Feature_imp <-}\StringTok{ }\KeywordTok{as.data.frame}\NormalTok{(}\KeywordTok{as.table}\NormalTok{(Feature_imp))}

\NormalTok{Feature_imp}\OperatorTok{$}\NormalTok{Var2 <-}\StringTok{ }\OtherTok{NULL}
\KeywordTok{names}\NormalTok{(Feature_imp) <-}\StringTok{ }\KeywordTok{c}\NormalTok{(}\StringTok{"Feature"}\NormalTok{, }\StringTok{"Importance"}\NormalTok{)}

\NormalTok{Feature_imp <-}\StringTok{ }\KeywordTok{head}\NormalTok{(Feature_imp[}\KeywordTok{order}\NormalTok{(}\OperatorTok{-}\NormalTok{Feature_imp}\OperatorTok{$}\NormalTok{Importance),],}\DecValTok{20}\NormalTok{)}

\KeywordTok{barplot}\NormalTok{(Feature_imp}\OperatorTok{$}\NormalTok{Importance, }\DataTypeTok{names =}\NormalTok{ Feature_imp}\OperatorTok{$}\NormalTok{Feature,}
        \DataTypeTok{xlab =} \StringTok{""}\NormalTok{, }\DataTypeTok{ylab =} \StringTok{"Importance"}\NormalTok{,}
        \DataTypeTok{main =} \StringTok{"Top 20 features"}\NormalTok{,}
        \DataTypeTok{las=}\DecValTok{2}\NormalTok{)}
\end{Highlighting}
\end{Shaded}

\includegraphics{Assignment_files/figure-latex/unnamed-chunk-9-1.pdf}

\paragraph{Finally we run the model on the test set in order to answer
the assignment
questions}\label{finally-we-run-the-model-on-the-test-set-in-order-to-answer-the-assignment-questions}

\begin{Shaded}
\begin{Highlighting}[]
\NormalTok{predict_FINAL <-}\StringTok{ }\KeywordTok{predict}\NormalTok{(mod_rf, test)}
\KeywordTok{print}\NormalTok{(predict_FINAL)}
\end{Highlighting}
\end{Shaded}

\begin{verbatim}
##  1  2  3  4  5  6  7  8  9 10 11 12 13 14 15 16 17 18 19 20 
##  B  A  B  A  A  E  D  B  A  A  B  C  B  A  E  E  A  B  B  B 
## Levels: A B C D E
\end{verbatim}


\end{document}
